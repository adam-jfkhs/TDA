% ==============================================================================
% SECTION 10: MACHINE LEARNING INTEGRATION
% Feedback applied:
% - Consistent ML guardrail sentence EVERYWHERE
% - Replace "profitable" with "positive risk-adjusted performance"
% - ONE authoritative table
% - Conservative AUC interpretation (≈0.52 is "barely above random")
% ==============================================================================

\section{Machine Learning Integration}
\label{sec:ml}

\subsection{Motivation}

Sections~\ref{sec:sector}--\ref{sec:variants} demonstrate that sector-specific topology achieves \positiverisk{} using simple threshold rules (75th percentile cutoffs for regime classification). However, these rules have critical limitations:

\begin{enumerate}
    \item \textbf{Binary classification}: Days are either ``stable'' or ``unstable'' with no gradation
    \item \textbf{Single feature}: Only H$_1$ volatility used, ignoring correlation dispersion and higher-order features
    \item \textbf{Fixed thresholds}: 75th percentile may be suboptimal or time-varying
\end{enumerate}

\textbf{Question}: Can machine learning extract topology-correlation signals more efficiently than rule-based thresholds?

\textbf{Critical framing}: \mlguardrail{} This section tests whether ML can improve \textit{regime classification}, not directional stock-picking.

\subsection{Methodology}

\subsubsection{Feature Engineering}

We construct 9 features per trading day combining topology and correlation statistics:

\paragraph{Topology Features (H$_0$ and H$_1$)}
\begin{enumerate}
    \item H$_0$ count (connected components)
    \item H$_0$ total persistence
    \item H$_1$ count (loops)
    \item H$_1$ mean persistence
    \item H$_1$ max persistence
    \item H$_1$ total persistence
    \item H$_1$ birth-death ratio
\end{enumerate}

\paragraph{Correlation Features}
\begin{enumerate}
    \item[8.] Mean pairwise correlation
    \item[9.] \textbf{Correlation dispersion} (standard deviation of pairwise correlations)
\end{enumerate}

All features calculated on 60-day rolling windows, aligned with strategy lookback period.

\subsubsection{Target Variable}

\textbf{Binary classification task}: Predict whether the next-day strategy return will be positive (class 1) or negative (class 0).

\textbf{Important caveat}: This is a \textit{regime detection} proxy, not pure directional prediction. Positive returns indicate topology correctly identified favorable regime; negative returns indicate unfavorable regime or signal noise.

\subsubsection{Models Tested}

\begin{enumerate}
    \item \textbf{TDA-Only Baseline}: Simple threshold rule (H$_1$ volatility $> p_{75}$ $\rightarrow$ unstable)
    \item \textbf{Random Forest} (RF): 100 trees, max depth 10, no hyperparameter optimization
    \item \textbf{Gradient Boosting} (GB): 100 estimators, learning rate 0.1, max depth 5
    \item \textbf{Neural Network} (NN): 2 hidden layers (16, 8 neurons), ReLU activation, Adam optimizer
\end{enumerate}

\textbf{Walk-forward split}: 70\% train (525 days), 30\% test (225 days), re-estimated every 252 days.

\subsection{Results}

\subsubsection{Model Performance: Improved Regime Classification but Weak Directional Prediction}

Table~\ref{tab:ml-authoritative} presents the authoritative comparison of all models tested.

% ==============================================================================
% AUTHORITATIVE TABLE: MACHINE LEARNING PERFORMANCE
% This is the SOURCE OF TRUTH for ML results
% ==============================================================================

\begin{table}[H]
\centering
\caption{Machine Learning Model Performance (Authoritative Results)}
\label{tab:ml-authoritative}
\begin{tabular}{@{}lccccccc@{}}
\toprule
\textbf{Model} & \textbf{$\boldsymbol{F_1}$ Score} & \textbf{AUC} & \textbf{Precision} & \textbf{Recall} & \textbf{Sharpe (net)} & \textbf{Status} \\
\midrule
\multicolumn{7}{l}{\textit{Baseline (Simple Threshold)}} \\
TDA-Only & 0.014 & 0.510 & 0.007 & 1.000 & $-0.56$ & Failed \\
\midrule
\multicolumn{7}{l}{\textit{Machine Learning Models}} \\
Random Forest & 0.512 & 0.519 & 0.489 & 0.537 & $+0.38$ & Success \\
Gradient Boosting & 0.547 & 0.521 & 0.521 & 0.574 & $+0.42$ & Success \\
Neural Network & \textbf{0.578} & \textbf{0.523} & \textbf{0.552} & \textbf{0.606} & $\mathbf{+0.47}$ & \textbf{Best ML} \\
\midrule
\textit{Improvement (NN vs Baseline)} & $+41\times$ & $+2.5\%$ & $+79\times$ & $-39\%$ & +1.03 & \\
\bottomrule
\end{tabular}

\vspace{0.2cm}

\begin{minipage}{\textwidth}
\footnotesize
\textbf{Notes:}
\begin{itemize}[leftmargin=*,noitemsep,topsep=0pt]
    \item All metrics calculated from walk-forward testing (70/30 train/test split, 225-day test periods)
    \item $F_1$ = harmonic mean of precision and recall: $F_1 = 2 \times \frac{\text{precision} \times \text{recall}}{\text{precision} + \text{recall}}$
    \item AUC = Area Under ROC Curve; random guessing = 0.5, perfect discrimination = 1.0
    \item \textbf{Critical interpretation}: AUC $\approx 0.52$ is \textcolor{red}{barely above random}, NOT ``good discrimination''
    \item Sharpe (net) = risk-adjusted returns after 5 basis points transaction costs
    \item \textbf{Compare to Table~\ref{tab:sector-authoritative}}: Sector-specific (simple thresholds) achieved Sharpe $+0.79$, outperforming ML
    \item \textbf{Data type:} \textcolor{blue}{EMPIRICAL} (real market data, features extracted from 2019-2024 price history)
    \item \textbf{Consistent guardrail}: These gains reflect improved regime classification rather than strong directional predictability
\end{itemize}
\end{minipage}

\end{table}


\textbf{Key Observations}:

\begin{enumerate}
    \item \textbf{TDA-only threshold catastrophically fails}: $F_1 = 0.014$, precision = 0.007
        \begin{itemize}
            \item Predicts nearly everything as ``unstable'' (recall = 1.0, precision $\approx 0$)
            \item Confirms simple thresholds are insufficient for signal extraction
        \end{itemize}

    \item \textbf{Machine learning dramatically improves $F_1$}: 0.014 $\rightarrow$ 0.578 (Neural Network)
        \begin{itemize}
            \item Represents 41$\times$ improvement in precision-recall balance
            \item All three ML models ($F_1 \in [0.51, 0.58]$) vastly outperform threshold baseline
        \end{itemize}

    \item \textbf{But AUC remains near random}: All models $\in [0.519, 0.523]$
        \begin{itemize}
            \item AUC = 0.5 is random guessing (coin flip)
            \item AUC $\approx 0.52$ is \textbf{barely above random}, not ``good discrimination''
            \item Consistent with efficient market limits on directional predictability
        \end{itemize}

    \item \textbf{Sharpe improvement exists but modest}: Neural Network achieves Sharpe $+0.47$ vs sector-specific baseline $+0.79$ (Table~\ref{tab:sector-authoritative})
        \begin{itemize}
            \item ML-based strategy underperforms simple sector-specific approach
            \item Suggests diminishing returns to complexity
        \end{itemize}
\end{enumerate}

\textbf{Conservative interpretation}: \mlguardrail{} The $F_1$ improvement reflects better identification of regime structure (when topology is informative vs noisy), but AUC $\approx 0.52$ confirms topology does \textit{not} provide strong directional alpha.

\subsubsection{Feature Importance: Correlation Dispersion Most Predictive}

Table~\ref{tab:feature-importance} ranks features by predictive importance (Neural Network model).

\begin{table}[H]
\centering
\caption{Feature Importance Rankings (Neural Network)}
\label{tab:feature-importance}
\begin{tabular}{@{}clcc@{}}
\toprule
\textbf{Rank} & \textbf{Feature} & \textbf{Importance} & \textbf{Category} \\
\midrule
1 & Correlation dispersion (std) & 21.3\% & Correlation \\
2 & H$_1$ mean persistence & 18.7\% & Topology (H$_1$) \\
3 & H$_1$ total persistence & 15.4\% & Topology (H$_1$) \\
4 & Correlation mean & 12.6\% & Correlation \\
5 & H$_1$ max persistence & 8.9\% & Topology (H$_1$) \\
6 & H$_1$ birth-death ratio & 7.3\% & Topology (H$_1$) \\
7 & H$_1$ count & 6.2\% & Topology (H$_1$) \\
8 & H$_0$ count & 5.8\% & Topology (H$_0$) \\
9 & H$_0$ persistence & 3.8\% & Topology (H$_0$) \\
\midrule
& \textbf{Topology features (total)} & \textbf{56.3\%} & \\
& \textbf{Correlation features (total)} & \textbf{33.9\%} & \\
\bottomrule
\end{tabular}
\end{table}

\textbf{Surprising finding}: \textbf{Correlation dispersion (std)} is the single most predictive feature (21.3\%), exceeding any individual topology metric.

\textbf{Interpretation}:
\begin{itemize}
    \item Periods with high correlation std (heterogeneous pairwise relationships) signal regime instability
    \item This validates Section~\ref{sec:sector}'s finding that correlation homogeneity is critical
    \item Topology features (56\% combined importance) add value \textit{beyond} correlations alone, but correlations remain foundational
\end{itemize}

\textbf{Practical implication}: Practitioners should monitor \textit{both} topology and correlation dispersion, not topology in isolation.

\subsection{Discussion}

\subsubsection{Machine Learning Validates Topology but Reveals Fundamental Limits}

\textbf{What ML confirms}:
\begin{enumerate}
    \item Topology contains regime information (not pure noise): $F_1$ improves 41$\times$ over random baseline
    \item Sector-specific approach's correlation-CV relationship (Section~\ref{sec:sector}) is learnable by ML
    \item Correlation dispersion is a critical complementary signal
\end{enumerate}

\textbf{What ML reveals about limits}:
\begin{enumerate}
    \item AUC $\approx 0.52$ indicates weak discrimination between favorable/unfavorable regimes
    \item Directional predictability remains near-random, consistent with efficient market hypothesis
    \item \mlguardrail{}
\end{enumerate}

\subsubsection{Comparison to Section 7 Sector-Specific Strategy}

\textbf{Sector-specific (simple thresholds)}: Sharpe $+0.79$, no ML required

\textbf{ML-based (Neural Network)}: Sharpe $+0.47$, added complexity

\textbf{Why does simple approach outperform ML?}
\begin{itemize}
    \item Sector-specific strategy \textit{pre-filters} for high-correlation regimes (exploits boundary condition $\rho > 0.5$)
    \item ML tries to learn regime classification from \textit{all} data (including low-correlation noise)
    \item Demonstrates value of domain knowledge (correlation homogeneity) over pure data-driven methods
\end{itemize}

\textbf{Potential hybrid}: Use ML for feature extraction \textit{within} pre-filtered sector-specific universes (not tested here, future work).

\subsubsection{Reconciliation with Section 7's Success}

Section~\ref{sec:sector} achieved Sharpe $+0.79$ using simple thresholds. This section shows ML-only achieves Sharpe $+0.47$. How to reconcile?

\textbf{Explanation}:
\begin{enumerate}
    \item Section 7 exploits \textbf{market segmentation} (compute topology per sector, filter $\rho > 0.5$)
    \item This section applies ML to \textbf{mixed data} (all sectors combined, including low-$\rho$ noise)
    \item The \textit{boundary condition} ($\rho > 0.5$) is more important than ML sophistication
\end{enumerate}

\textbf{Implication}: Architectural design (when to compute topology, where to apply filters) dominates model choice.

\subsubsection{Practical Recommendations}

Based on these results:

\paragraph{For Practitioners}
\begin{itemize}
    \item \textbf{Start with correlation filtering}: Only compute topology when mean $\rho > 0.5$
    \item \textbf{Monitor correlation dispersion}: std($\rho$) > threshold may signal regime instability
    \item \textbf{Use ML for refinement, not replacement}: ML can improve $F_1$ within viable regimes but won't overcome fundamental limits (AUC $\approx 0.52$)
\end{itemize}

\paragraph{For Researchers}
\begin{itemize}
    \item \textbf{Feature engineering matters more than model choice}: Random Forest, Gradient Boosting, Neural Network all perform similarly ($F_1 \in [0.51, 0.58]$)
    \item \textbf{Topology provides incremental value}: 56\% feature importance (Table~\ref{tab:feature-importance}) beyond correlations (34\%)
    \item \textbf{Weak AUC is informative, not disappointing}: Confirms regime detection use case rather than pure alpha
\end{itemize}

\subsection{Limitations}

\begin{enumerate}
    \item \textbf{No hyperparameter optimization}: Models use default parameters to avoid overfitting; tuned models might achieve $F_1 \sim 0.6$--0.65 but unlikely to improve AUC meaningfully
    \item \textbf{Binary classification simplification}: Regime intensity (degree of stability) might be better modeled as regression, not classification
    \item \textbf{Limited ensemble testing}: Only tested individual models; stacking or blending could improve performance
    \item \textbf{Feature set incomplete}: Could add technical indicators (RSI, Bollinger bands) or fundamental factors (P/E ratios)
\end{enumerate}

\subsection{Conclusion}

Machine learning improves regime classification ($F_1$ increases 41$\times$, from 0.014 to 0.578) but reveals fundamental limits on directional predictability (AUC $\approx 0.52$, barely above random). \mlguardrail{}

\textbf{Key insight}: Correlation dispersion (std) is the most predictive single feature (21\% importance), validating Section~\ref{sec:sector}'s emphasis on correlation homogeneity. Topology features add incremental value (56\% combined importance) but cannot overcome efficient market limits on directional alpha.

\textbf{Practical takeaway}: Use topology for \textit{risk overlays} (dynamic exposure scaling based on regime classification), not standalone return generation.

This addresses Research Question 3 (\textit{Can ML extract topology signals efficiently?}): \textbf{Yes for regime classification ($F_1$ improves), but no for directional prediction (AUC $\approx 0.52$)}. The boundary conditions identified in Section~\ref{sec:sector} (correlation homogeneity, $\rho > 0.5$) remain more important than ML sophistication.
