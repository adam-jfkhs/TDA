\section{Baseline Results}
\label{sec:results}

\subsection{Walk-Forward Validation Performance}

Walk-forward out-of-sample testing reveals severe underperformance relative to preliminary claims. Table~\ref{tab:performance-summary} summarizes performance across all tested strategy variations.

\begin{table}[h]
\centering
\caption{Performance Summary Across Strategy Variations}
\label{tab:performance-summary}
\begin{tabular}{@{}lccc@{}}
\toprule
\textbf{Strategy} & \textbf{Sharpe Ratio} & \textbf{CAGR} & \textbf{Max DD} \\
\midrule
TDA - Equities (OOS) & $-0.56$ & $-13.55\%$ & $-34.68\%$ \\
TDA - Alternatives & $-1.87$ & $-22.52\%$ & $-44.28\%$* \\
Simple MR - Equities & $-1.58$ & $-25.85\%$ & $-48.12\%$* \\
Simple MR - Alternatives & $-2.08$ & $-22.73\%$ & $-52.45\%$* \\
\bottomrule
\end{tabular}
\end{table}

*Maximum drawdown values estimated based on cumulative return patterns. TDA - Equities represents actual calculated maximum drawdown.

All tested variations produced negative returns. The topology-filtered strategy on equities performed least poorly, losing 13.55\% annually with a Sharpe ratio of $-0.56$.

\begin{table}[h]
\centering
\caption{Statistical Significance of Performance Metrics}
\label{tab:statistical-significance}
\begin{tabular}{@{}lccccc@{}}
\toprule
\textbf{Strategy} & \textbf{Sharpe} & \textbf{95\% CI} & \textbf{t-stat} & \textbf{p-value} & \textbf{Cohen's d} \\
\midrule
TDA - Equities & $-0.56$ & $[-0.64, -0.48]$ & $-14.3$ & $<0.001$ & 0.90 \\
TDA - Alternatives & $-1.87$ & $[-2.01, -1.73]$ & $-25.3$ & $<0.001$ & 1.60 \\
Simple MR - Equities & $-1.58$ & $[-1.71, -1.45]$ & $-23.7$ & $<0.001$ & 1.49 \\
Simple MR - Alternatives & $-2.08$ & $[-2.24, -1.92]$ & $-26.3$ & $<0.001$ & 1.66 \\
\bottomrule
\end{tabular}
\end{table}

\textit{Interpretation:} All strategies significantly underperform (negative Sharpe, $p < 0.001$); results robust across asset classes and parameter variations. TDA filter reduces but does not eliminate losses.

Table~\ref{tab:statistical-significance} presents statistical significance testing for all strategies. Standard errors calculated using analytical formulas adjusted for return non-normality (Lo, 2002). All strategies exhibit statistically significant negative Sharpe ratios ($p < 0.001$), indicating the underperformance is not attributable to random sampling variation. The 95\% confidence intervals for all strategies exclude zero, confirming systematic rather than stochastic failure. Effect sizes (Cohen's $d > 0.8$ for all strategies) indicate large practical significance beyond statistical significance.

\subsection{Topology Regime Detection Analysis}

The topology-based regime classifier successfully identified historical crisis periods, including the COVID crash (March 2020), the 2022 Federal Reserve tightening cycle, and 2024 AI-driven volatility. Visual inspection confirms that unstable periods align with known market stress events.

However, the filter exhibited critical failure during the 2022 test period, classifying 0\% of days as unstable despite sustained market decline. This suggests the topology signal lags major regime shifts, detecting instability only after significant damage has occurred.

Despite this limitation, topology filtering provided measurable value: the TDA strategy achieved Sharpe $-0.56$ versus $-1.58$ for simple mean reversion without topological filtering, representing approximately 50\% reduction in losses.

\subsection{Parameter Sensitivity}

Systematic parameter sweep across 12 combinations revealed universal negative performance. The absence of any positive-return configuration indicates fundamental rather than parametric failure.

Notable patterns observed:
\begin{itemize}
\item Shorter lookbacks (40 days) marginally less negative
\item Higher correlation thresholds (0.4) produce worst results due to sparser networks
\item Position count has minimal impact
\end{itemize}

The universal underperformance across parameters rules out simple optimization fixes.
