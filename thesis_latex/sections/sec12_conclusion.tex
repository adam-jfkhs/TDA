\section{Conclusion}
\label{sec:conclusion}

\subsection{Summary of Findings}

\textit{[Content from: thesis\_expansion/SECTION\_12\_CONCLUSION.md]}

\subsection{What I Would Do Next}

\textbf{Feedback applied: One-paragraph design insights (not more experiments)}

If continuing this research, I would focus on three architectural directions:

\paragraph{1. Multi-Horizon Regime Classifiers}
Rather than binary stable/unstable classification, develop a continuous ``regime intensity'' score combining daily topology volatility (short-term) with monthly eigenvalue concentration (long-term). This would enable gradual position scaling rather than binary cash/invested decisions.

\paragraph{2. Fiedler Value as Real-Time Proxy}
Section~\ref{sec:theory} demonstrates Fiedler value (graph Laplacian $\lambda_2$) correlates $\rho = -0.99$ with topology CV while computing 50× faster. For intraday applications, replacing persistent homology with Fiedler-based regime detection could enable real-time risk monitoring without computational bottlenecks.

\paragraph{3. Topology as Feature Selector for Multi-Strategy Portfolios}
Rather than using topology to generate standalone trading signals, deploy it as a \textit{meta-filter} for existing strategies: scale exposure to momentum/value/quality factors based on topological regime stability. This leverages topology's strength (regime detection) while avoiding its weakness (weak directional prediction).

\textbf{Common thread}: These extensions recognize topology's value lies in \textit{structural framework design}, not incremental alpha generation.

