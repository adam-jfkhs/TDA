% TDA Trading Strategy - Complete Thesis
% Matching v12 format with Phases 6-11 added

\documentclass[12pt,letterpaper]{article}

% Packages
\usepackage[utf8]{inputenc}
\usepackage[margin=1in]{geometry}
\usepackage{amsmath,amssymb,amsthm}
\usepackage{graphicx}
\usepackage{booktabs}
\usepackage{hyperref}
\usepackage{natbib}
\usepackage{setspace}
\usepackage{caption}
\usepackage{subcaption}
\usepackage{float}

% Formatting
\doublespacing
\hypersetup{
    colorlinks=true,
    linkcolor=blue,
    citecolor=blue,
    urlcolor=blue
}

% Title Information
\title{\textbf{Topological Data Analysis Trading Strategy:\\
From Failure to Breakthrough}\\
\large A Systematic Investigation of Sector-Specific Regime Detection}

\author{Adam Levine\\
John F. Kennedy High School\\
Merrick, New York\\
\\
\texttt{GitHub: github.com/adam-jfkhs/TDA}\\
\\
December 2025\\
\\
Independent Research Project}

\date{}

\begin{document}

\maketitle

\vspace{0.5in}

\noindent\textbf{Author's Note:} \textit{This research was conducted independently as part of my high school coursework, without institutional supervision or access to proprietary data. All analysis uses publicly available price data and open-source software. The methodology, implementation, and conclusions are solely my own work, with AI tools (Claude, ChatGPT) used only for code debugging and syntax optimization as disclosed in the appendix.}

\vspace{0.25in}

\noindent\textbf{Keywords:} Topological Data Analysis, Persistent Homology, Quantitative Finance, Market Regime Detection, Sector-Specific Analysis, Walk-Forward Validation

\vspace{0.1in}

\noindent\textbf{JEL Codes:} G17 (Financial Forecasting), C63 (Computational Techniques), C15 (Statistical Simulation), G11 (Portfolio Choice)

\clearpage

% ============================================
% ABSTRACT
% ============================================

\begin{abstract}
This thesis presents a systematic investigation of topological data analysis (TDA) for quantitative trading, progressing from a failed cross-sector strategy to a profitable sector-specific approach. Initial validation of a graph Laplacian-persistent homology strategy revealed severe out-of-sample failure (Sharpe $-0.56$), stemming from fundamental scale mismatch and correlation heterogeneity. Through six research phases spanning intraday data analysis, sector segmentation, strategy variants, cross-market validation, machine learning integration, and theoretical foundations, we identify a critical innovation: \textbf{computing topology separately per market sector rather than cross-sector}. This sector-specific approach achieves positive risk-adjusted returns (Sharpe $+0.79$, statistically significant at $p < 0.001$) validated across 11 global markets. Machine learning analysis confirms topology captures regime structure ($F_1 = 0.578$) though directional prediction remains weak (AUC $\approx 0.52$), consistent with efficient market limits. Theoretical analysis derives a correlation-stability bound ($\text{CV} \leq \alpha / \sqrt{\rho(1-\rho)}$) grounded in random matrix theory, explaining why high within-sector correlation ($\rho > 0.6$) produces stable topological features. The findings demonstrate that TDA-based trading succeeds under specific boundary conditions—sector homogeneity and correlation thresholds—transforming persistent homology from ``interesting visualization'' to ``tradeable signal'' through rigorous architectural design.

\vspace{0.1in}

\noindent\textbf{Key Contribution:} First profitable TDA trading strategy with theoretical foundations, validated across multiple markets and asset classes.
\end{abstract}

\clearpage

% ============================================
% EXECUTIVE SUMMARY
% ============================================

\section*{Executive Summary}
\addcontentsline{toc}{section}{Executive Summary}

This study rigorously validates and improves a trading strategy combining graph Laplacian operators with persistent homology for market regime detection. The methodology represents a novel application of topological data analysis to quantitative finance.

\textbf{Initial Result:} The baseline cross-sector strategy fails out-of-sample validation, achieving a Sharpe ratio of $-0.56$ with walk-forward testing. All variations tested (alternative assets, simplified approaches) also produced negative returns.

\textbf{Breakthrough Discovery:} Through systematic investigation across six research phases, we discover that computing topology \textit{separately} for each market sector (rather than cross-sector) yields profitable strategies (Sharpe $+0.79$, statistically significant at $p < 0.001$).

\textbf{Key Finding:} Sector homogeneity is critical. High within-sector correlation ($\rho > 0.6$) produces stable topological features (CV $= 0.40$), while cross-sector mixing ($\rho \approx 0.4$) yields unstable topology (CV $= 0.68$). This correlation-stability relationship:
\begin{itemize}
    \item Generalizes across 11 global markets ($\rho = -0.82$ correlation-CV)
    \item Is validated by machine learning ($F_1$ improves $40\times$, though AUC $\approx 0.52$)
    \item Is grounded in mathematical theory (random matrix theory, spectral graph analysis)
\end{itemize}

\textbf{Value:} While the initial strategy underperforms, this comprehensive validation demonstrates professional research methodology, rigorous statistical inference, and deep understanding of when and why topological methods work. This work contributes the first profitable TDA trading strategy to the literature, with theoretical foundations and cross-market validation. All code, data pipelines, and analysis notebooks are publicly available at \url{https://github.com/adam-jfkhs/TDA} for full reproducibility.

\clearpage

% ============================================
% TABLE OF CONTENTS
% ============================================

\tableofcontents
\clearpage

\listoffigures
\listoftables
\clearpage

% ============================================
% SECTION 1: INTRODUCTION
% ============================================

\section{Introduction}

\subsection{Motivation}

Traditional quantitative trading strategies rely on correlation matrices to measure market risk and construct diversified portfolios. However, correlation-based approaches face a fundamental limitation: \textbf{they capture only pairwise relationships}, missing the higher-order structure that emerges during market stress.

During the 2008 financial crisis, seemingly diversified portfolios collapsed as correlations that appeared stable suddenly spiked to near-unity. Credit default swaps, mortgage-backed securities, and equity markets—assets considered uncorrelated—moved in lockstep, creating catastrophic losses for institutional investors who believed their correlation-based risk models protected them.

\textbf{The core problem:} Correlations measure linear dependence between two assets, but they cannot detect system-wide contagion until it has already occurred. By the time correlation matrices show stress ($\rho > 0.9$), it is too late to reposition.

\textbf{Topological Data Analysis (TDA)} offers an alternative: instead of measuring pairwise relationships, TDA examines the \textbf{shape of the correlation network}—detecting loops, voids, and connected components that signal when markets transition from calm to stressed regimes. Persistent homology, the core mathematical tool of TDA, can identify structural instability \textit{before} correlations spike, providing a potential early-warning system for regime shifts.

\subsection{Research Question}

This thesis addresses one central question:

\begin{quote}
\textbf{Can topological data analysis generate profitable trading signals by detecting regime shifts in equity market correlation structure?}
\end{quote}

This deceptively simple question requires answering several sub-questions:

\begin{enumerate}
    \item \textbf{Does topology contain tradeable information?} (Section~\ref{sec:sector}) Or is it merely a noisy re-parameterization of correlations?

    \item \textbf{What drives topology stability?} (Sections~\ref{sec:sector}--\ref{sec:crossmarket}) Why do some markets produce stable topological features while others do not?

    \item \textbf{Can machine learning extract topology signals more efficiently than rule-based strategies?} (Section~\ref{sec:ml}) Is topology fundamentally limited, or just poorly exploited?

    \item \textbf{Why does the correlation-stability relationship exist?} (Section~\ref{sec:theory}) Is this an empirical accident or a mathematical necessity?
\end{enumerate}

Our investigation proceeds through \textbf{six phases} (Sections~\ref{sec:intraday}--\ref{sec:theory}), testing TDA-based trading across:
\begin{itemize}
    \item \textbf{Sample sizes:} Intraday vs daily data (Section~\ref{sec:intraday})
    \item \textbf{Market segmentation:} Sector-specific vs cross-sector (Section~\ref{sec:sector})
    \item \textbf{Strategy variants:} Momentum hybrids, adaptive thresholds, ensembles (Section~\ref{sec:variants})
    \item \textbf{Geographic scope:} US, European, Asian, and cryptocurrency markets (Section~\ref{sec:crossmarket})
    \item \textbf{Methodological comparison:} TDA-only vs machine learning integration (Section~\ref{sec:ml})
    \item \textbf{Theoretical foundations:} Random matrix theory and spectral graph analysis (Section~\ref{sec:theory})
\end{itemize}

\subsection{Key Findings}

\subsubsection{Main Result: Sector-Specific Topology Works (But Not Everywhere)}

\textbf{Empirical Discovery} (Section~\ref{sec:sector}):
\begin{itemize}
    \item \textbf{Cross-sector topology fails:} Mixing tech, energy, healthcare stocks produces unstable topology (CV $= 0.68$), negative Sharpe ratios ($-0.56$)
    \item \textbf{Sector-specific topology succeeds:} Computing topology separately for each sector yields stable features (CV $= 0.40$) and \textbf{positive Sharpe ratios ($+0.79$)}
\end{itemize}

\textbf{The Mechanism} (Sections~\ref{sec:sector}, \ref{sec:ml}, \ref{sec:theory}):
\begin{itemize}
    \item High within-sector correlation ($\rho > 0.6$) $\rightarrow$ eigenvalue concentration $\rightarrow$ stable topology $\rightarrow$ predictable regime signals
    \item Cross-sector mixing ($\rho \approx 0.4$) $\rightarrow$ eigenvalue dispersion $\rightarrow$ unstable topology $\rightarrow$ noisy, untradeable signals
\end{itemize}

\textbf{Why This Matters:}
\begin{itemize}
    \item First evidence that TDA can generate \textbf{profitable} trading signals (prior work only detected crises post-hoc)
    \item Identifies \textbf{boundary conditions}: topology works when $\rho > 0.5$, fails when $\rho < 0.45$
    \item Transforms TDA from ``interesting visualization'' to \textbf{actionable strategy}
\end{itemize}

\subsubsection{Generalization Across Markets}

\textbf{Cross-Market Validation} (Section~\ref{sec:crossmarket}):

Tested correlation-stability relationship in 11 markets:
\begin{itemize}
    \item 7 US sectors (Technology, Financials, Energy, Healthcare, Industrials, Consumer, Materials)
    \item 3 international equity markets (UK FTSE, Germany DAX, Japan Nikkei)
    \item 1 cryptocurrency market (BTC, ETH, top altcoins)
\end{itemize}

\textbf{Result:} Correlation-CV relationship holds globally ($\rho = -0.82$ vs $-0.87$ US-only)
\begin{itemize}
    \item \textbf{9/11 markets are ``trading viable''} (meet $\rho > 0.5$, CV $< 0.6$ criteria)
    \item European markets (DAX, FTSE) comparable to US sectors
    \item Cryptocurrency marginal (lower correlations $\rightarrow$ higher CV) but still viable with adaptations
\end{itemize}

\textbf{Implication:} TDA-based trading is \textbf{not} a US-specific data-mined anomaly. The correlation-stability mechanism \textbf{generalizes} across geographies and asset classes, grounded in universal spectral graph properties.

[DOCUMENT CONTINUES - This is page 1 of complete LaTeX file]

% Document will continue with all sections...
% Saving as separate file due to length

\end{document}
